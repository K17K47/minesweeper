\documentclass[10pt,a4paper]{report}
\usepackage[utf8]{inputenc}
\usepackage[portuguese]{babel}
\usepackage[T1]{fontenc}
\usepackage{amsmath}
\usepackage{amsfonts}
\usepackage{amssymb}
\usepackage{graphicx}

\usepackage{titlesec}
\usepackage{termcal}

\graphicspath{{./img/}}

\author{Lucas Lopes Costa}
\title{Relatório}

\titleformat{\chapter}{\normalfont\huge}{\thechapter.}{20pt}{\huge\bf}

\begin{document}
\chapter{Resumo}

Neste trabalho desenvolvemos uma versão simplificada de um jogo de computador bastante conhecido, o Campo Minado. Tendo em vista a finalidade didática da execução deste projeto, buscamos utilizá-lo como plataforma para pôr em prática variadas técnicas e ferramentas de planejamento e desenvolvimento de software. O projeto foi executado por uma equipe de quatro alunos no período de 2 de junho à 1 de julho.

\chapter{Introdução}

Neste trabalho desenvolvemos uma versão simplificada de um jogo de computador bastante conhecido, o \textbf{Campo Minado}. Tal jogo, conhecido também como \textbf{Minesweeper}, tem origens na década de 60, e consiste de um tabuleiro, inicialmente coberto, onde aleatóriamente são semeadas bombas, e demarcadas dicas numéricas à respeito do número de bombas nas redondezas de uma casa. O jogador deve a cada jogada indicar uma casa, e marcá-la como minada, ou ``pisar'' sobre a mesma, revelando seu conteúdo e, se vazia, de sua vizinhança. O objetivo é demarcar a localização de todas as bombas presentes no tabuleiro, utilizando as dicas reveladas, sem que se ``pise'' sobre uma casa minada.

Tendo em vista a finalidade didática da execução de tal projeto, buscamos utilizá-lo como plataforma para o exercício de variadas práticas em planejamento e desenvolvimento de software. Com isso estabelecemos um particionamento de etapas e tarefas na equipe, seccionando o projeto em módulos, cada um sob a responsabilidade de um integrante. Debatemos e definimos também um conjunto específico de ferramentas a ser utilizado, visando facilitar a comunicação e o desenvolvimento em grupo.

Dividiu-se a execução do projeto em três fases principais, \textbf{Planejamento}, \textbf{Desenvolvimento} e \textbf{Teste}.

\chapter{Planejamento}
Nesta primeira fase, que representa o início do projeto como um todo, o foco esteve em definir as bases para as fases de desenvolvimento e teste, o que inclui a determinação de um \textbf{Cronograma}, a determinação de uma \textbf{Estrutura do Projeto}, a divisão das tarefas de desenvolvimento, e a escolha das \textbf{Ferramentas} em software utilizadas pela equipe.
\section{Cronograma}
Iniciando no dia 2 de junho, organizamos a primeira reunião da equipe de projeto, onde determinamos os primeiros tópicos a serem determinados, e a partir destes formulamos o cronograma apresentado abaixo.
\begin{calendar}{6/1/16}{5}
\setlength{\calboxdepth}{.3in}

\calday[Quarta-feira]{\classday}
\calday[Quinta-feira]{\classday}
\calday[Sexta-feira]{\classday}
\skipday\skipday
\calday[Segunda-feira]{\classday}
\calday[Terça-feira]{\classday}

\caltexton{1}{Início da Fase de Planejamento}
\caltexton{2}{Reunião:\\Cronograma e estrutura do projeto}
\caltexton{4}{Reunião:\\Estrutura do projeto e ferramentas}
\caltexton{5}{Reunião:\\Estrutura do projeto}
\caltexton{6}{Reunião:\\Estrutura do projeto}
\caltexton{9}{Início da Fase de Desenvolvimento}
\caltexton{10}{Reunião:\\Escrita de código em grupo}
\caltexton{18}{Primeiro Build funcional}
\caltexton{19}{Início da Fase de Teste}
\caltexton{22}{Conclusão do Projeto}
\end{calendar}
\section{Estrutura do Projeto}
Durante as primeiras reuniões um dos principais tópicos foi a Estrutura do Projeto como um todo. Já que a idéia dominante nas discussões era a de como seria a secção do projeto e a divisão das tarefas de desenvolvimento, e já que este determina como será escrito o software todo, este mostrou-se como o tópico mais importante de toda a primeira fase da execução do projeto.

Logo no início, ficara decidido que o projeto teria uma estrutura modular, com unidades de código bem delimitadas, e relacionadas entre si por interfaces pré-definidas, de maneira que o código seja o mais flexível possível, mantendo ainda uma alta independência de implementação e uma relativa simplicidade de uso dos módulos.

Estas escolhas de design direcionaram o projeto para um estilo próximo à Programação Orientada à Objetos e a Programação Modular. Com isso, as posteriores reuniões foram reservadas apenas à determinação dos módulos e de suas interfaces de código. Ao fim do planejamento chegamos a uma organização composta por 3 módulos: \textbf{Mecânica de Jogo}, \textbf{Interface de Usuário} e \textbf{Estruturas de Dados}.

\begin{figure}[h]
\centering
\includegraphics[width=0.5\textwidth]{diagrama.png}
\caption{Módulos e Arquivos do Projeto}
\end{figure}

\subsection{Estruturas de Dados}
Neste módulo são implementadas estruturas de dados auxiliares, como vetores para coordenadas em 2-D, e estruturas de dados básicas do jogo, como o tabuleiro do campo minado, e suas interfaces de programação, tal como métodos para criação e destruição destas, e métodos para manipulação de seus estados internos e dados armazenados.

\subsubsection{Estrutura ``vec''}
Esta estrutura é uma simples estrutura para armazenamento de coordenadas cartesianas 2-D inteiras no primeiro quadrante. Não possui métodos implementados.

\begin{tabular}{| c | c | l |}
\hline Nome da estrutura & \multicolumn{2}{c |}{vec}\\
\hline Nome do Campo & Tipo & Descrição\\
\hline x & \textbf{unsigned} & Componente cartesiana em $x$\\
y & \textbf{unsigned} & Componente cartesiana em $y$\\
\hline
\end{tabular}

\subsubsection{Estrutura ``jogada\_t''}
Esta estrutura tem a função de servir como estrutura de dados auxiliar, transportando o código e coordenada de uma jogada realizada pelo usuário. Possui métodos {\it getters} e {\it setters}.\\\\
\begin{tabular}{| c | c | l |}
\hline Nome da estrutura & \multicolumn{2}{c |}{jogada\_t}\\
\hline Nome do Campo & Tipo & Descrição\\
\hline v & \textbf{vec} & Coordenada cartesiana do ponto da jogada\\
opjogada & \textbf{char} & Código da jogada\\
\hline
\end{tabular}

\paragraph{Métodos}
\begin{itemize}
\item \textbf{char} {\it get\_jogada}(\textbf{jogada\_t} tab) : Retorna o código da jogada armazenada em uma estrutura \textbf{jogada\_t}.
\item \textbf{void} {\it set\_jogada}(\textbf{jogada\_t*} tab, \textbf{char} op) : Armazena um código de jogada em uma estrutura \textbf{jogada\_t}.
\item \textbf{vec} {\it get\_coord}(\textbf{jogada\_t} jogada) : Retorna a coordenada armazenada em uma estrutura \textbf{jogada\_t}.
\item \textbf{void} {\it set\_coord}(\textbf{jogada\_t*} jogada, \textbf{unsigned} x, \textbf{unsigned} y) : Armazena uma coordenada em uma estrutura \textbf{jogada\_t}.
\end{itemize}

\subsubsection{Estrutura ``tabuleiro\_t''}
Nesta estrutura são armazenadas informações sobre o tabuleiro do campo minado, tais como suas dimensões, número de minas semeadas, e dois vetores armazenando o estado atual do tabuleiro, uma máscara das casas livres, e informações sobre as minas e dicas de suas localizações. Possui métodos para inicialização, destruição, {\it getters} e {\it setters}.\\\\
\begin{tabular}{| c | c | l |}
\hline Nome da estrutura & \multicolumn{2}{c |}{tabuleiro\_t}\\
\hline Nome do Campo & Tipo & Descrição\\
\hline m & \textbf{unsigned} & Número de linhas do tabuleiro \\
n & \textbf{unsigned} & Número de colunas do tabuleiro\\
q & \textbf{unsigned} & Número de minas semeadas\\
usr & \textbf{char*} & Vetor máscara do tabuleiro\\
gabarito & \textbf{char*} & Vetor das minas e dicas do tabuleiro \\
\hline
\end{tabular}

\paragraph{Métodos}
\subparagraph{Inicialização e Destruição}
\begin{itemize}
\item \textbf{void} {\it inicializa\_tab}(\textbf{tabuleiro\_t*} tab) : Aloca espaço para os vetores da estrutura \textbf{tabuleiro\_t} de acordo com as dimensões armazenadas.
\item \textbf{void} {\it destroi\_tab}(\textbf{tabuleiro\_t*} tab) : Libera o espaço alocado para os vetores da estrutura \textbf{tabuleiro\_t}.
\end{itemize}
\subparagraph{\textit{Getters}}
\begin{itemize}
\item \textbf{unsigned} {\it get\_m}(\textbf{tabuleiro\_t} tab) : Retorna a dimensão vertical do tabuleiro.
\item \textbf{unsigned} {\it get\_n}(\textbf{tabuleiro\_t} tab) : Retorna a dimensão horizontal do tabuleiro.
\item \textbf{unsigned} {\it get\_q}(\textbf{tabuleiro\_t} tab) : Retorna o número de minas do tabuleiro.
\item \textbf{char} {\it get\_usr}(\textbf{tabuleiro\_t} tab, \textbf{vec} v) : Retorna o conteúdo de uma determinada coordenada \textbf{v} do tabuleiro máscara.
\item \textbf{char} {\it get\_gabarito}(\textbf{tabuleiro\_t} tab, \textbf{vec} v) : Retorna o conteúdo de uma determinada coordenada \textbf{v} do tabuleiro das minas e dicas.
\end{itemize}
\subparagraph{\textit{Setters}}
\begin{itemize}
\item \textbf{void} {\it set\_m}(\textbf{tabuleiro\_t*} tab, \textbf{unsigned} m) : Armazena a dimensão vertical do tabuleiro.
\item \textbf{void} {\it set\_n}(\textbf{tabuleiro\_t*} tab, \textbf{unsigned} n) : Armazena a dimensão horizontal do tabuleiro.
\item \textbf{void} {\it set\_q}(\textbf{tabuleiro\_t*} tab, \textbf{unsigned} q) : Armazena o número de minas do tabuleiro.
\item \textbf{void} {\it set\_usr}(\textbf{tabuleiro\_t*} tab, \textbf{vec} v, \textbf{char} usr) : Armazena um caractere \textbf{usr} em uma determinada coordenada \textbf{v} do tabuleiro máscara.
\item \textbf{void} {\it set\_gabarito}(\textbf{tabuleiro\_t*} tab, \textbf{vec} v, \textbf{char} gabarito) : Armazena um caractere \textbf{gabarito} em uma determinada coordenada \textbf{v} do tabuleiro das minas e dicas.
\end{itemize}
\subsection{Mecânica de Jogo}
Neste módulo são implementadas funções relacionadas ao funcionamento do jogo, tal como rotinas para processar condições de vitória, ou rotinas para determinar os resultados de uma jogada efetuada pelo usuário. Não possui estruturas próprias. Depende do módulo \textbf{Estruturas de Dados}.
\subsubsection{Método CriarTabuleiroUsr}
\paragraph{Sintaxe}
\textbf{void} {\it CriarTabuleiroUsr}(\textbf{tabuleiro\_t*} usr);
\subsubsection{Método ColocarBombas}
\paragraph{Sintaxe}
\textbf{void} {\it ColocarBombas}(\textbf{tabuleiro\_t*} gabarito);
\subsubsection{Método AvaliarVizinhos}
\paragraph{Sintaxe}
\textbf{void} {\it AvaliarVizinhos}(\textbf{tabuleiro\_t*} gabarito);
\subsubsection{Método Revela}
\paragraph{Sintaxe}
\textbf{void} {\it Revela}(\textbf{tabuleiro\_t*} gabarito, \textbf{vec} v);
\subsubsection{Método ExecutaJogada}
\paragraph{Sintaxe}
\textbf{void} {\it ExecutaJogada}(\textbf{tabuleiro\_t*} gabarito, \textbf{jogada\_t*} jogada, \textbf{int*} p);
\subsubsection{Método ConfereRevelados}
\paragraph{Sintaxe}
\textbf{int} {\it ConfereRevelados}(\textbf{tabuleiro\_t} tab);
\subsection{Interface de Usuário}
T
\section{Ferramentas}
T
\chapter{Desenvolvimento}
T
\section{Execução}
T
\chapter{Conclusão}
T
\end{document}