\documentclass[10pt,a4paper]{report}
\usepackage[utf8]{inputenc}
\usepackage[portuguese]{babel}
\usepackage[T1]{fontenc}
\usepackage{amsmath}
\usepackage{amsfonts}
\usepackage{amssymb}
\usepackage[usenames,dvipsnames,svgnames,table]{xcolor}
\author{Lucas Lopes Costa}
\title{Interfaces}
\begin{document}
T
\chapter{Mech - Game Mechanics}
T
\chapter{UI - User Interface}
T
\chapter{Data - Data Structures}
T
\section{Estruturas}
T
\subsection{Tabuleiro}
\begin{tabular}{| c | c | l |}
\hline Nome da estrutura & \multicolumn{2}{c |}{tabuleiro\_t}\\
\hline Nome do Campo & Tipo & Descrição\\
\hline m & \textbf{\color{green}unsigned} & Número de linhas do tabuleiro \\
n & \textbf{\color{green}unsigned} & Número de colunas do tabuleiro\\
q & \textbf{\color{green}unsigned} & Número de minas semeadas\\
usr & \textbf{\color{red}char*} & Vetor máscara do tabuleiro\\
gabarito & \textbf{\color{red}char*} & Vetor \\
\hline
\end{tabular}

\subsubsection{Métodos}
T

\subsection{Vetor}
\begin{tabular}{| c | c | l |}
\hline Nome da estrutura & \multicolumn{2}{c |}{vec}\\
\hline Nome do Campo & Tipo & Descrição\\
\hline x & \textbf{\color{green}unsigned} & Componente cartesiana em $x$\\
y & \textbf{\color{green}unsigned} & Componente cartesiana em $y$\\
\hline

\end{tabular}

\subsubsection{Métodos}
{\bf Não há métodos para esta estrutura}

\subsection{Jogada}
\begin{tabular}{| c | c | l |}
\hline Nome da estrutura & \multicolumn{2}{c |}{jogada\_t}\\
\hline Nome do Campo & Tipo & Descrição\\
\hline v & \textbf{\color{blue}vec} & Coordenada cartesiana do ponto da jogada\\
opjogada & \textbf{\color{red}char} & Código da jogada\\
\hline
\end{tabular}

\subsubsection{Métodos}
\begin{itemize}
\item \textbf{{\color{red}char} {\color{purple}get\_jogada}({\color{orange}jogada\_t} tab)} : Retorna o código da jogada armazenada em uma estrutura \textbf{\color{orange}jogada\_t}.
\item \textbf{void {\color{purple}set\_jogada}({\color{orange}jogada\_t*} tab, {\color{red}char} op)} : Armazena um código de jogada em uma estrutura \textbf{\color{orange}jogada\_t}.
\end{itemize}
\end{document}